\documentclass[12pt, titlepage]{article}

\usepackage{booktabs}
\usepackage{tabularx}
\usepackage{hyperref}
\hypersetup{
    colorlinks,
    citecolor=black,
    filecolor=black,
    linkcolor=red,
    urlcolor=blue
}
\usepackage[round]{natbib}

\usepackage{graphicx}
\usepackage{paralist}
\usepackage{amsfonts}
\usepackage{amsmath}
\usepackage{hhline}
\usepackage{booktabs}
\usepackage{multirow}
\usepackage{multicol}
\usepackage[normalem]{ulem}
\usepackage{xcolor}

\oddsidemargin 0mm
\evensidemargin 0mm
\textwidth 160mm
\textheight 200mm
\renewcommand\baselinestretch{1.0}

\pagestyle {plain}
\pagenumbering{arabic}

\newcounter{stepnum}

\usepackage{color}

\newif\ifcomments\commentstrue

\ifcomments
\newcommand{\authornote}[3]{\textcolor{#1}{[#3 ---#2]}}
\newcommand{\todo}[1]{\textcolor{red}{[TODO: #1]}}
\else
\newcommand{\authornote}[3]{}
\newcommand{\todo}[1]{}
\fi

\newcommand{\wss}[1]{\authornote{blue}{SS}{#1}}

\title{SE 3XA3: MIS\\OpenCameraRefined}

\author{Team \#211, CAMERACREW
		\\ Faisal Jaffer, jaffem1
		\\ Dominik Buszowiecki, buszowid
		\\ Pedram Yazdinia, yazdinip
		\\ Zayed Sheet, sheetz
}

\date{\today}


\begin{document}

\maketitle

\pagenumbering{roman}
% \tableofcontents
% \listoftables
% \listoffigures

\begin{table}[hp]
\caption{\bf Revision History}
\begin{tabularx}{\textwidth}{p{3cm}p{2cm}X}
\toprule {\bf Date} & {\bf Version} & {\bf Notes}\\
\midrule
March 14, 2020 & 1.0 & Initial document\\
\midrule
April 4, 2020 & 2.0 & Initial document\\
\bottomrule
\end{tabularx}
\end{table}

\newpage

\section* {Gesture Controller Module}

\subsection*{Module}

GestureController

\subsection* {Uses}

TensorflowObjectDetectionAPI \\
ClassifierConstants \\
CameraController \\
Filter \\
Classifier

\subsection* {Syntax}

\subsubsection* {Exported Constants}

N/A

\subsubsection* {Exported Types}

GestureController = ?

\subsubsection* {Exported Access Programs}

\begin{tabular}{| l | l | l | l |}
\hline
\textbf{Routine name} & \textbf{In} & \textbf{Out} & \textbf{Exceptions}\\
\hline
GestureController &$  $ &  GestureController & classifier\_initialize\_failure\\
\hline
processImage &$  $ &  & \\
\hline
setFrame & $byte$[] &  & \\
\hline
captureImage &  &  & \\
\hline
showFilter &  &  & \\
\hline
\end{tabular}

\subsection* {Semantics}

\subsubsection* {State Variables}

imageFrame: $byte []$  \texit{\# current camera frame} \\
classifier: $Classifier $ \texit{\# TensforFlow model}\\
filter: $Filter $ \\
recognition: $Recognition[]$ \texit{\# classification on the current frame}\\
\subsubsection* {Environment Variables}

None

\subsubsection* {State Invariant}

None

\subsubsection* {Assumptions}

\begin{itemize}
\item The mathematical operator $\backslash$ represents integer division. For example $ 8 \backslash 5 = 1$.
\end{itemize}

\subsubsection* {Access Routine Semantics}

\noindent GestureController():
\begin{itemize}
\item transition: $classifier, filter:= Classifier(ClassifierConstants.getInferenceName(),\break ClassifierConstants.getLabelName()), Filter()$
\item output: $out := self$\\
\item exception: $exc := ((ClassifierConstants.getInferenceName() == null | \break ClassifierConstants.getLabelName() == null) \Rightarrow classifier\_initialize\_failure)\\$

\end{itemize}

\noindent processImage():
\begin{itemize}
\item transition: $recognition := classifier.recognizeImage(bitmapFromByte(imageFrame)) \Rightarrow \break (recognition.title == ClassifierConstants.Smile \rightarrow captureImage() | recognition.title == ClassifierConstants.Thumb \rightarrow filter.changeFilter())$
\item output: None\\
\item exception: None\\

\end{itemize}

\noindent setFrame($byte$[] data):
\begin{itemize}
\item transition: imageFrame:= data
\item output: None\\
\item exception: None\\
\end{itemize}

\noindent captureImage():
\begin{itemize}
\item transition: $CameraController.captureImage()$\\
\item output: None\\
\item exception: None\\
\end{itemize}

\noindent showFilter():
\begin{itemize}
\item transition: None\\
\item output: None\\
\item exception: None\\
\end{itemize}

\subsection*{Local Functions}


\noindent bitmapFromByte($byte$[] data):
\begin{itemize}
\item transition: Matrix mat := ImageUtils.convertYUV420SPToARGB8888(data)
\item output: $out := $ mat\\
\item exception: None\\
\end{itemize}


\medskip
\medskip


\newpage

\section* {Recognition Module}

\subsection*{Module}

Recognition

\subsection* {Uses}

RectF

\subsection* {Syntax}

\subsubsection* {Exported Constants}

N/A

\subsubsection* {Exported Types}

Recognition = ?

\subsubsection* {Exported Access Programs}

\begin{tabular}{| l | l | l | l |}
\hline
\textbf{Routine name} & \textbf{In} & \textbf{Out} & \textbf{Exceptions}\\
\hline
Recognition &$ String, String, \mathbb{Q}, RectF $ &  Recognition & \\
\hline
getID &$  $ & $String$ & \\
\hline
getTitle &$ $ & $String$ & \\
\hline
getConfidence &$  $ & $\mathbb{Q}$ & \\
\hline
getLocation &$ Bitmap $ & $RectF$ & \\
\hline
\end{tabular}

\subsection* {Semantics}

\subsubsection* {State Variables}

id: $String$ \texit{\# unique id assignment since multiple recognitions possible} \\ 
title: $String$ \texit{\# label of the recognition}\\
confidence: $\mathbb{Q}$ \texit{\# confidence level of the classification}\\
location: RectF \texit{\# pixel location of the recognition }\\

\subsubsection* {Environment Variables}

None

\subsubsection* {State Invariant}

None

\subsubsection* {Assumptions}

\begin{itemize}
\item Invalid arguments will not be provided into the Recognition and getLocation routines.
\end{itemize}

\subsubsection* {Access Routine Semantics}

\noindent Recognition(id, title, confidence, location):
\begin{itemize}
\item transition: $ id, title, confidence,location := id, title, confidence, location$
\item output: $out := self$ \\
\item exception:None \\

\end{itemize}

\noindent getID():
\begin{itemize}
\item transition: None\\
\item output: $out := id$\\
\item exception: None\\

\end{itemize}

\noindent getTitle():
\begin{itemize}
\item transition: None\\
\item output: $out := title$\\
\item exception: None\\

\end{itemize}

\noindent getConfidence():
\begin{itemize}
\item transition: None\\
\item output: $out := confidence$\\
\item exception: None\\

\end{itemize}

\noindent getLocation():
\begin{itemize}
\item transition: None\\
\item output: $location$\\
\item exception: None\\

\end{itemize}

\subsection*{Local Functions}

None

\medskip
\medskip


\newpage

\section* {Classifier Module}

\subsection*{Module}

Classifier

\subsection* {Uses}

Recognition  \\
ClassifierConstants \\
Bitmap \\
TF \\

\subsection* {Syntax}

\subsubsection* {Exported Constants}

N/A

\subsubsection* {Exported Types}

None

\subsubsection* {Exported Access Programs}

\begin{tabular}{| l | l | l | l |}
\hline
\textbf{Routine name} & \textbf{In} & \textbf{Out} & \textbf{Exceptions}\\
\hline
Classifier &$ String, String, (\mathbb{Z}, \mathbb{Z}) $ & Classifier & classifier\_initialize\_failure\\
\hline
recognizeImage &$ Bitmap $ & $Recognition$ & \\
\hline
\end{tabular}

\subsection* {Semantics}

\subsubsection* {State Variables}

modelFilename: $String$ \texit{\# link to the model inference }\\
labelFilename: $String$ \texit{\# link to the model labels }\\
inputSize: $(\mathbb{Z}, \mathbb{Z})$ \texit{\# size of model input} \\
model: TF.model\texit{\# stored model} \\

\subsubsection* {Environment Variables}

None

\subsubsection* {State Invariant}

None

\subsubsection* {Assumptions}

\begin{itemize}
\item Invalid arguments will not be provided to the Classifier and recognizeImage routines.
\end{itemize}


\subsubsection* {Access Routine Semantics}

\noindent  Classifier(modelFilename, labelFilename, inputSize):
\begin{itemize}
\item transition: $model := new TF.model(modelFilename, labelFilename)$
\item output: $out := self$\\
\item exception: $exc := new TF.model(modelFilename, labelFilename) == null \break \Rightarrow (classifier\_initialize\_failure)\\$

\end{itemize}

\noindent recognizeImage(b):
\begin{itemize}
\item transition: None\\
\item output: $out := model.detect(b)$\\
\item exception: None\\

\end{itemize}

\subsection*{Local Functions}

None

\medskip
\medskip


\newpage

\section* {Classifier Constants Module}

\subsection*{Module}

ClassifierConstants

\subsection* {Uses}


\subsection* {Syntax}

\subsubsection* {Exported Constants}

inferenceName: $String$ \\
labelName: $String$

\subsubsection* {Exported Types}

None

\subsubsection* {Exported Access Programs}

\begin{tabular}{| l | l | l | l |}
\hline
\textbf{Routine name} & \textbf{In} & \textbf{Out} & \textbf{Exceptions}\\
\hline
getInferenceName & & $String$ & \\
\hline
getLabelName & & $String$ & \\
\hline
\end{tabular}

\subsection* {Semantics}

\subsubsection* {State Variables}


\subsubsection* {Environment Variables}

None

\subsubsection* {State Invariant}

None

\subsubsection* {Assumptions}

None

\subsubsection* {Access Routine Semantics}

\noindent getInferenceName():
\begin{itemize}
\item transition: None\\
\item output: $out := $ inferenceName\\
\item exception: None\\

\end{itemize}

\noindent getLabelName():
\begin{itemize}
\item transition: None\\
\item output: $out := $ labelName\\
\item exception: None\\

\end{itemize}

\subsection*{Local Functions}

None

\medskip
\medskip


\newpage

\section* {Filter Module}

\subsection*{Module}

Filter

\subsection* {Uses}

Filter Constants

\subsection* {Syntax}

\subsubsection* {Exported Constants}

N/A


\subsubsection* {Exported Types}

N/A

\subsubsection* {Exported Access Programs}

\begin{tabular}{| l | l | l | l |}
\hline
\textbf{Routine name} & \textbf{In} & \textbf{Out} & \textbf{Exceptions}\\
\hline
changeFilter &  &  & \\
\hline
getFilter &  & $\mathbb{Z}$ & \\
\hline
\textcolor{red}{setFrame} & \textcolor{red}{$byte[]$} &  & \\
\hline
\textcolor{red}{processImage} & &  & \\
\hline
\textcolor{red}{getFiltered} & & \textcolor{red}{$Bitmap$} & \\
\hline

\end{tabular}

\subsection* {Semantics}

\subsubsection* {State Variables}

filterIndex : $\mathbb{Z}$ \textit{\#Which index in the FILTERS constant from the filters module is selected}
\textcolor{red}{
imageFrame: $byte []$  \texit{\# current camera frame} \\
rgbFrameBitmap: $Bitmap$ \texit{\# bitMap version of camera frame}\\
filtered: $Bitmap$ \texit{\# Filtered version of rgbFrameBitmap } \\
}

\subsubsection* {Environment Variables}

\textcolor{red}{preview: \textit{\# Android Camera Preview object}}

\subsubsection* {State Invariant}

$\text{filterIndex} < |FILTERS|$

\subsubsection* {Assumptions}

The $\%$ operator represents the mathematical modulus operator

\subsubsection* {Access Routine Semantics}

changeFilter():
\begin{itemize}
\item transition: $\text{filterIndex} := (\text{filterIndex} + 1) \% |FILTERS + 1|$
\item output: None

\end{itemize}

\noindent getFilter():
\begin{itemize}
\item transition: None
\item output: $out := \text{filterIndex}$
\end{itemize}

\noindent \textcolor{red}{setFrame($byte$[] data):}
\begin{itemize}
\item \textcolor{red}{transition: imageFrame:= data}
\item \textcolor{red}{output: None}
\item \textcolor{red}{exception: None}
\end{itemize}

\noindent \textcolor{red}{processImage():}
\begin{itemize}
\item \textcolor{red}{transition: \\filterIndex = 0 $\implies$ filtered := $null$ \\
filterIndex\ \!= 0 $\implies$ filtered := setFiltered(rgbFrameBitmap)
}

\item \textcolor{red}{output: None}
\item \textcolor{red}{exception: None}
\end{itemize}

\noindent \textcolor{red}{getFiltered():}
\begin{itemize}
\item \textcolor{red}{transition: None}
\item \textcolor{red}{output: $out$ := filtered}
\end{itemize}

\newpage

\section* {Filter Constants Module}

\subsection*{Module}

Constants

\subsection* {Uses}

OpenCV

\subsection* {Syntax}

\subsubsection* {Exported Constants}

GRAYSCALE: \sout{($\mathbb{Z}$[], $\mathbb{Z}$[])} \textcolor{red}{ColorMatrixColorFilter}\\
RED\_FILTER: \sout{($\mathbb{Z}$[], $\mathbb{Z}$[])} \textcolor{red}{ColorMatrixColorFilter} \\
BLUE\_FILTER: \sout{($\mathbb{Z}$[], $\mathbb{Z}$[])} \textcolor{red}{ColorMatrixColorFilter}\\
\sout{\textit{\# The values of the filters will be obtained from the OpenCV lookup tables. They are 2xN matrices which map a source colour of a pixel to a final colour after applying a filter. The value of N depends on the total number of possible colours}}\\
\textit{\# The values of the filters are obtained from code snippets found online, 
the type ColorMatrixColorFilter is a type built into android}\\
FILTERS = [GRAYSCALE, RED\_FILTER, BLUE\_FILTER]\\

\subsubsection* {Exported Types}

None

\subsubsection* {Exported Access Programs}

None

\subsection* {Semantics}

\subsubsection* {State Variables}

None

\subsubsection* {State Invariant}

None

\newpage



\end {document}

