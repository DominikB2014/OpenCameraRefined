\documentclass[12pt, oneside]{article}   	
\usepackage{geometry}                		
\geometry{letterpaper}                   		
\usepackage{graphicx}				
\usepackage{amssymb, amsmath}
\usepackage{cite}
\usepackage{color}



\title{SFWR 3XA3: Problem Statement}
\author{Group 211
		\\ Faisal Jaffer
		\\ Pedram Yazdinia
		\\ Dominik Buszowiecki
		\\ Zayed Sheet
		\\ Due: January 24, 2020
		\\Professor: Dr. Asghar Bokhari
		\\ Lab: L02}
\date{}

\begin{document}
\maketitle

Many people using their camera application on their Android mobile devices encounter issues where a certain functionality may be inaccessible. This could be because the functionality is difficult to find, use or simply does not exist. Often, we can see that users tend to prefer external applications such as Snapchat, Instagram and TikTok for a camera because of the features and functionalities that make it more accessible towards the user’s needs. While these external applications each have their own unique feature that can fulfill a specific user requirement, they still do not provide a well rounded solution to the problem. 
\\

OpenCamera is an open source camera application that allows users to capture photos outside of the stock Android camera app by providing a simpler UI. However, this software does not include many of the key functionalities that a user may want from their camera. Our goal is to modify the Open Camera application in a way to entice more users to use the application for all their capturing needs while maintaining a clean user interface. 
\\

The stakeholders include any user that requires accessible features in their camera application, as well as a clean UI for their day-to-day capturing needs. More specifically, the stakeholders devices are limited to OpenCamera compatible versions of Android (4.0 and later).

    

\bibliographystyle{acm}
\bibliography{ref.bib}

\end{document}  